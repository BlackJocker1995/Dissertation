短暂的博士生涯即将结束。
从硕士一路走来收获颇多,相识的各位在学术成长上给予了我很大的帮助。
首先我需要感谢这几年指导过我的教师马建峰教授、杨超教授和马卓教授,他们在研究方向上给予了我很多的指导。
其次我要感谢校外指导老师马思奇与李卷儒老师,他们在我学术写作和学术规划上提供了非常大的帮助。

再者我要感谢同门各位老师的关照(排名不分先后):李兴华、沈玉龙、姜奇、张俊伟、孙聪、习宁、卢笛、马鑫迪
、李腾、冯鹏斌、张嘉伟、魏大卫、马卓然、王翔宇、刘洋。

感谢并祝福与我一同科研的学弟(妹)和师兄(姐),愿我们的前途一同光明。

最后,感谢在百忙之中评审此论文的专家评委。