%% 无人机背景和现状
随着航空技术的不断进步和数字化时代的到来,无人机作为一种不需要搭载人员直接操纵的飞行器应运而生。
其在工业和生活领域的应用迅速普及,可用于协助各个商业部门、政府组织、货物运输等核心职能方面。
由于无人机无需在线驾驶员,且体积小、速度快,常被用于执行难以到达或人力成本高昂的任务。
但是,同样由于小型化的问题,无人机的硬件的运算能力和软件可靠性设计无法与传统航空飞行器相比较。
特别是系统设计层面,不同系统之间缺乏统一的规范,缺少传统系统中的保障机制,导致其系统设计层面的可靠性机制不够完善,存在一定的设计缺陷。
这些缺陷影响无人机不同的功能模块,使得在实际运行过程中该模块的运行偏离其原始设计初衷。
这些设计缺陷可能在特定的场景、配置或环境下影响无人机的任务执行,甚至造成设备损坏。
因此,确保无人机系统的可靠性具有重要的意义。

%% 无人机存在的可靠性问题
针对上述问题,本文结合人工智能的相关技术,针对目前无人机系统设计中存在的三个影响可靠性的问题,
开展相应的研究并提供了对应的解决方案。
其中研究的问题包括,无人机安全检查模块的问题,无人机参数配置模块的问题,无人机姿态受影响下的恢复问题。
具体的研究成果如下:

\begin{enumerate}
    \item %% 安全检查不可靠的解决方案
    无人机依靠安全检查模块来识别不安全的状态,例如侧倾,动力损失等。
    它使用条件规则来验证系统内部的特定动态数值来表明不安全的状态是否发生,从而启动相应的补救措施。
    例如,在动力损失状态下提升电机的转速从而提供更多的动力。
    然而,较为复杂的不安全事件(例如阵风和轻微碰撞)会影响安全模块的判断,从而造成漏报和误报的风险,导致人身伤害、财产损失或硬件损坏。
    
    针对无人机安全检查模块处理复杂的不安全事件时能力受限的情况。
    文章提出了一种辅助检查模块,能够增强原系统的判断能力。
    该模块结合无人机飞行状态的变化属性和机器学习技术,建立了一个状态不变量预测器,用于预测未来的飞行状态变化。
    辅助检查模块利用该预测器为飞行数据建立特征描述。
    最终,辅助检查模块适用识别器检测这些复杂的不安全事件,并制定相应的正确应对策略。
    新集成的辅助可以补充无人机模块检查未能检测到的遗漏事件,以及减少由于原先检测系统过于敏感而造成的误报。
    在流行的无人机飞行控制系统下进行测试表明,该方案在处理遗漏事件时将准确率从2\%提高到81\%,在处理误报时,误报率从44\% 降至6\%。

    \item %% 配置的解决方案
    开发商和制造商为飞行控制程序提供可配置的控制参数,以支持各种环境和任务,并提供这些参数的建议范围,以确保飞行安全。
    然而,这种灵活的机制也会带来一种被称为范围规范错误(Range Specification Bugs)的漏洞。
    即使配置中的每个参数均设置在官方的建议范围内,参数值组合也可能影响无人机的物理稳定性,造成坠毁,卡死,偏航等问题。
    
    针对无人机系统参数配置机制面临的范围规范错误的漏洞。
    文章给出了一种预测引导的模糊测试方案用于发掘该漏洞,该方案通过多轮测试搜会触发漏洞的配置,并提出更合理的建议范围以剔除这些配置。
    其采用元启发式模糊测试方法来寻搜索这些配置。
    比较特别的是,该方案采用了机器学习的预测器来对搜索过程中的配置进行预测评估。
    这样可以减少模糊测试中的执行验证次数,提高了搜索效率。
    在流行的无人机飞行控制系统下进行测试表明,该系统成功地报告了范围规范错误的配置。

    \item %% 配置在线修复方案 
    受到范围规范错误的影响,无人机的姿态无人机会在飞行过程中出现异常。
    表现为逐渐增大的飞行数据误差和难以纠正的姿态晃动,上述模糊测试方案能解决起飞前的配置问题。
    而对于飞行中的配置改变以及主动攻击效果无法应对。
    
    针对无人机在飞行中受攻击或上传错误配置而导致的姿态不稳定问题。
    文章给出一种自动的实时检测和纠正方案,使无人机能够自我调整飞行稳定性,应对飞行过程中的配置变化或者外部主动攻击。
    该方案通过观察偏差数据(实际飞行状态与预期飞行状态之间的差距)来检测无人机的稳定性。
    当报告不稳定性时,该方案立即调用经过预先训练的智能模型自动生成适当的修正,然后重新配置无人机以避免进入失控状态。
    在流行的无人机飞行控制系统下进行测试表明,该方案能够消除实验中85\%由配置造成的不稳定。
    
\end{enumerate}




