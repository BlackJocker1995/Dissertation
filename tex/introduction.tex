\section{研究背景及意义}
% 无人机的废话
得益于人工智能等许多领域的进步,无人机在我们的日常生活中的应用场景迅速增长。
从军事战术到商业应用,无人机在各个领域展现出了巨大的潜力~\cite{chabot2018trends,banos2020assessment,patino2009adaptive,shukla2016application,huang2021multi,li2018uav}。
例如,新闻报道和电影制作可以通过无人机获取不同的视角和高质量的影像,极大地丰富了视觉效果。
在地理测绘方面,无人机可以到达人类难以抵达的地方,为地理信息系统提供了更多的数据来源~\cite{news1}。
在灾害情况信息收集方面,无人机可以在灾难发生后迅速进入现场,为救援团队提供准确的现场信息~\cite{news2}。
在人道主义救援行动中,无人机可以通过热传感器寻找被困的人,从而提供及时的救援~\cite{polka2017use}。
此外,亚马逊Prime Air、谷歌Wing、UPS Flight Forward等公司公开宣布了计划从空中运送包裹的快递配送方案~\cite{news3}。
这些公司的无人机配送服务都是在尝试利用无人机技术来提高配送效率、降低配送成本,并尝试为顾客提供更快速的配送服务。
东京奥运会开幕式等活动也以无人机灯光秀为特色。

% 在军事方面,无人机可以执行侦察、打击等任务,降低了士兵面临的风险。
% 在物流领域,无人机可以用于快递送货,加快了货物的运输速度。
% 在科研领域,无人机可以用于大气探测、地质勘探等任务,为科学研究提供了新的工具。
% 在娱乐领域,无人机也被广泛应用于航拍、比赛等活动,为人们带来了全新的娱乐体验。
% 总的来说,其多功能性使其成为一种重要工具,随着技术不断发展和创新,无人机在各个领域的作用将继续扩大。

% 无人机的而可靠性不行
然而,由于无人机轻量化、高实时性、多应用场景的特性,大部分无人机系统(如\tool{Ardupilot}和\tool{PX4})的实现较为初级,缺乏一定的可靠性和功能验证,其造成的系统漏洞在部分场景和条件下会对无人机产生物理损伤。
且由于无人机体系架构碎片化的问题 (例如四旋翼和六旋翼),现有的开发和使用人员难以根据自身飞行任务的需要,准确的配置或修改无人机。
近几年的研究~\cite{wang2021exploratory,choi2020cyber}表明,无人机系统在其设计与实现层面存在较大的问题,容易引入可被利用的系统设计漏洞。
甚至全球最大的商用无人机开发商大疆,也存在旧机型上的系统设计漏洞~\cite{schiller2023drone},且并未被修复。
这些碎片化、多场景、复杂的飞行问题更加剧了无人机系统的设计难度,进而在无人机系统上引入系统可靠性的问题。

% 现有的发现的无人机的问题
这些可靠性问题存在于无人机系统中的不同机制模块当中,都会影响无人机的飞行。
例如,由于系统设计缺乏抗干扰而数据二次验证,无人机传感器模块读数受到干扰或者其融合模块噪声影响时~\cite{daneshmand2012low,gaspar2020capture,huang2015gps},会使得无人机系统错误认知自己所处的飞行环境,进而做出错误的反应,引发飞行失控。
由于缺乏基本的加密和通信验证机制,无人机的通信模块如果受到恶意伪造攻击时,可能会接收错误的命令和消息~\cite{li2018protecting,chen2019machine},同样会导致无人机飞行失控。
更难以察觉的是系统功能设计层面的问题。
功能设计缺陷~\cite{rvfuzzer,han2022control,wang2021exploratory,choi2020cyber,aggarwal2020path}会使无人机中相应的系统影响飞行中的可靠性和稳定性。
在特定环境、配置或任务的情况下,这些影响可能会造成无人机失控,完全脱离使用者的掌控。
同时,由于是系统功能设计所导致,因此此类问题并不容易被开发者提前发现,进而难以定位问题根源,无法及时修复。
此外,如果无人机的使用者缺乏操作经验,也会造成主动触发这些问题,从而\dquote{主动}影响无人机的系统可靠性。


% 现有的解决和仍然存在的问题
相关综述~\cite{derhab2023internet,omolara2023drone}表明,无人机系统中大量影响飞行可靠性的问题
与姿态控制和任务执行相关。
也就是说,一个系统功能与姿态控制和任务执行的联系越紧密,其功能设计上问题越容易导致无人机出现不可挽回的损失。
例如,无人机系统中的数据融合功能影响系统对自身姿态的判断、安全检查功能影响无人机自身的姿态调整、参数配置功能影响无人机的姿态调整和任务适应性、姿态稳定功能则直接影响飞行时无人机的动态调整。
现有大量的研究~\cite{heredia2008sensor,quinonez2020savior,samy2008neural,fei2018cross,dash2021stealthy,sindhwani2020unsupervised,lu2017motor}针对数据融合模块提出了切实可行的可靠性提升方案,有效减少数据对无人机飞行的影响。
然而,无人机其他功能仍面临着较初级的保护。
无人机不一致现象~\cite{choi2020cyber}揭示了其安全检查功的设计缺陷;
参数规范错误~\cite{rvfuzzer,han2022control}则指出了无人机参数配置功能的功能缺陷;
而部分研究~\cite{choi2020software,dash2021pid}直接表明,无人机缺乏最基本的飞行中动态修复保护方案。
也就是说,无人机系统中的部分核心组件仍需要提升其功能的可靠性。
本文需要关注无人系统的特点,通过整合先进的人工智能和机器学习技术,面向上述无人机系统中仍存在的系统问题,研究更加可靠的安全检查、参数配置和在线修复方案,从而实现更加可靠的飞行,最大程度地减少意外事件和损失的发生。


% 近年来的相关安全研究也更加聚焦无人机系统。
% 通过模糊测试~\cite{rvfuzzer,han2022control}、故障定位~\cite{mayday,choi2020cyber,kim2022pgpatch}、异常检测~\cite{heredia2008sensor, crowdgps,lu2017motor}等手段强化无人机系统的方案也日益增多。
% 整体上,无人机安全性和可靠性增强通常从三个角度入手。


% \begin{enumerate}
%     \item \textbf{增强模块功能:} 无人机自身提供了初级的安全模块用于应对飞行中出现的安全问题或不稳定飞行。
%     但是开发者由于缺乏经验,此类模块所提供的功能较为初级,无法覆盖更广的场景。
%     因此,部分研究~\cite{heredia2008sensor,choi2020cyber}通过修改其自身的安全模块来提升无人机的能力。
%     此类研究通过提升无人机应对已有问题的能力来提高安全性和可靠性。

%     \item \textbf{完善系统机制:} 无人机系统设计环节可能存在漏洞。
%     因此部分研究~\cite{rvfuzzer,han2022control}通过算法搜索逻辑漏洞从而修复无人机系统中存在的问题。
%     此类研究通过完善系统机制来提高安全性和可靠性。

%     \item \textbf{增加动态功能:} 无人机本身是一种动态飞行设备,许多问题出现在无人机的飞行过程当中。
%     部分研究~\cite{choi2020software,dash2021pid}通过在无人机上或监测系统中添加额外的功能来及时发现无人机的问题并提供相应的补救措施,从而在不中断无人机飞行的情况下,修正无人机。
%     此类研究通过提供额外的在线监控和修补功能提高安全性和可靠性。
% \end{enumerate}
% 然而,相关综述~\cite{derhab2023internet,omolara2023drone}表明,现阶段的无人机系统仍会出不可预料安全问题。

% 其无人机的系统的安全性和可靠性仍有加强的空间。
% 因此,本文需要关注无人系统的特点,通过整合先进的技术,丰富无人机安全相关的功能,提升无人机系统自身的安全性和可靠性,保证飞行和任务的正常执行,从而最大程度地减少意外事件和损失的发生。


\section{国内外研究现状}
本节将介绍国内外与无人机系统可靠性相关的研究。
包含数据融合功能中异常数据识别的研究,无人机系统漏洞发掘及修复相关研究,以及有助于系统漏洞发掘的现有工具和方案。


\subsection{无人机异常数据识别}
现有的文献从多角度对无人机上的错误识别和异常检测方法进行了讨论。

\subsubsection{无人机异常数据监控}
基于监测和参数估计的方法是最常用的故障检测方法。 
近年来发表的大多数关于无人机系统的工作也使用基于监测的方法。
监测的方法背后的基本思想是通过在确定性设置中使用观察器来估计系统的输出。
一旦预估与实际发生偏离,则认为是异常。

一些研究用不变量来作为飞行过程中数据的监测器来识别异常。
\citet{choi2018detecting}提出了一种攻击检测框架,通过推导和监控控制不变量 (Control Invariant) 来实时识别针对无人机车辆的外部物理攻击。
该方法通过对车辆的物理特性、其控制算法和物理定律进行联合建模来提取此类不变量的方法。
这些不变量以状态空间形式表示,然后可以实现并插入车辆的控制程序二进制文件以进行运行时不变检查。
一旦飞行值偏离控制不变量所预期的值过大时,该程序认为无人机受到了攻击。
为了检测无人机系统中的故障组件,故障检测方法~\cite{heredia2008sensor} 构建了一个监测器模型,它根据历史输入和输出估计无故障条件下的输出。
它采用输出值与预测值之间的偏差来识别有故障的传感器和执行器组件。
\citet{quinonez2020savior}通过攻击检测算法、反击模型以及在硬件最小性能开销这三个功能的组合,引入了\tool{SAVIOR},一种隐形攻击监测系统,基于物理数据的监测机制为自动驾驶汽车提供安全保障,与控制不变量不同,它们的不变量的设计是参考的转子的转速输出。

其他研究采用冗余数据的方式来构建监测器。
基于神经网络的验证方法~\cite{samy2008neural} 利用飞行参数之间的分析冗余来检测传感器故障。
它在无人机模型上利用飞行参数之间的分析冗余,选择使用扩展最小资源分配网络 (Extended Minimum Resource Allocating Network) 算法在线训练的径向基函数 (Radial-Basis Function) 进行神经网络建模,用生成的模型去检测错误。
\citet{fei2018cross} 设计了一个名为\tool{BlueBox}的框架,通过外部件来监控车辆减少了原始系统的修改。
它从车辆源代码上提取车辆的控制逻辑,并在外部硬件上实现了与控制逻辑相类似的功能从而实现高精度的错误检测。 
结合软件和硬件冗余,\tool{BlueBox}能够检测对传感器、控制器、车辆动力学、执行器和控制器操作系统的各种攻击。

此类数据监控方式从无人机数据变化模式下手,建立数据变化的\dquote{基本规律},以模型的形式预测无人机的正确输出,从而实现检测。

\subsubsection{机器学习异常检测}
在此进一步的基础上,一些方案尝试将数据监控与机器学习内容相结合。
\citet{dash2021stealthy} 展示了基于控制的入侵检测中的漏洞,并提出了三种逃避检测和破坏无人机任务的隐形攻击方法。 
针对这种攻击,该同时提供了执行攻击的自动化算法,无需攻击者花费大量精力或了解飞行设备的具体细节。
\citet{sindhwani2020unsupervised}使用机器学习方法和模型通过对无人机的传感器飞行参数进行训练来生成一种新颖的异常检测框架。
它引入了聚类方法来解决群体飞行中标记数据的缺乏和正常数据与异常数据之间的不平衡问题。
该框架基于学习的数据驱动方法,用于群体飞行中的异常检测。
\citet{lu2017motor} 使用强化学习进行运动异常检测。
他们采用学习模型来预测电机温度的变化趋势,并监控每个动态状态的预测温度偏差。
如果检测到温度异常,无人机将自动降落,如果温度恢复正常,则恢复飞行,这个系统可以防止无人机在极端环境下坠毁。
自动电机异常检测系统提高了安全性,允许更积极地使用自动驾驶,从而减少因通信丢失和控制器错误引起的碰撞。
\citet{ahn2019learning}提出了一种基于学习的数据驱动方法,用于群体飞行中的异常检测。
该方法主要试图解决群体飞行中缺乏标记数据以及正常和异常数据不平衡的问题。
他们首先利用主成分分析首先降低飞行数据时间序列的维数,然后使用K均值聚类将数据分组并将它们标记为四类。 
最终,该方法训练了一个由一维卷积神经网络和多层感知器串联而成的深度神经网络,来提取特征和进行检测异常。
\citet{abbaspour2016detection} 介绍了一种自适应神经网络算法来检测无人机设备中的故障数据注入攻击。
他们使用该网络来检测无人机传感器中的注入故障,其中扩展卡尔曼滤波(Extended Kalman Filter,EKF) 用于在线调整神经网络权重。
这种在线调整使攻击检测更快、更准确。

类似的,本文问题的解决方法将机器学习方法与监测器相结合,实现了比传统线性方案更高的预测准确度。



\subsection{基于学习的模糊测试}
随着机器学习理论的发展,其应用于安全测试的案例也随之增多。
\tool{NEUZZ}~\cite{she2019neuzz}介绍了一种新颖、高效、可扩展的程序平滑技术,该技术使用前馈神经网络,可以逐步学习复杂的、真实世界的程序分支行为的平滑近似,即预测由特定输入行使的目标程序的控制流边缘。
该技术进一步提出了一种梯度引导的搜索策略,计算并利用平滑近似的梯度来确定目标突变的位置,使目标程序中检测到的错误数量最大化。
\citet{kim2021pgfuzz}开发了\tool{PGFUZZ},一种基于策略的模糊测试框架用于验证无人机程序是否遵循定义的安全和功能策略。
\tool{PGFUZZ}包括,预处理,策略引导的模糊测试,和Bug后处理共三个组件。
在预处理组件中,\tool{PGFUZZ}通过度量时间逻辑的表示的策略来表达无人机中的操作逻辑,并通过查找与测试策略相关的输入来最小化模糊测试空间。
然后,策略引导的模糊测试对预处理组件识别的输入进行变异,当全局距离变为负值时,就认为检测到违反策略。
最终,\tool{PGFUZZ}通过错误后处理组件来排除与策略违规无关的输入来最小化触发错误的输入序列。
\tool{ExploitMeter}~\cite{yan2017exploitmeter}提出了一个名为\tool{ExploitMeter}的框架,它使用动态模糊测试和基于机器学习的预测来更新对软件可利用性的判断。
\tool{ExploitMeter}框架的核心是一个模仿人类评估者认知过程的贝叶斯推理引擎:使用其静态特征,评估者首先从基于机器学习的预测中得出它对软件系统可利用性的先验信心,然后通过使用不同的模糊器进行一些动态模糊测试,并通过新观察来更新她对软件可利用性的信念。
总体上说,\tool{ExploitMeter}的推理引擎可以被描述为一个动态非线性系统,其输入为来自基于机器学习的预测和动态模糊测试。

学习引导的模糊测试~\cite{chen2019learning}提出了一种自动化的机器学习引导方法,用于构建工业系统中网络攻击的模糊测试。
该技术使用预测机器学习模型和元启发式搜索来智能地模糊执行器命令,并系统地将系统驱动到不同类别的不安全物理状态。
它首先通过对描述其正常行为的物理数据日志进行机器学习算法训练,学习数据变化模型。
学习到的模型可以用来预测当前的物理状态在不同的执行器配置下会如何演变。
随后,它通过网络对执行器进行模糊测试,从而找到驱动系统进入目标不安全状态的攻击序列。
\citet{bottinger2018deep}研究了如何将模糊测试形式化为学习问题。
对于模糊测试来说,模糊器生成新的输入,然后使用每个输入运行目标程序。 
对于每个程序执行,模糊器提取运行时信息以评估当前输入的质量。
同样,强化学习设置定义了与系统交互的代理。 
每个执行的动作都会导致系统的状态转换。
在执行每个动作后,代理会观察下一个状态并获得奖励。 
两者在工作流程上是相似的,因此该研究最终采用Q-learning~\cite{mnih2013playing}算法来进行模糊测试。
\tool{DeepSmith}~\cite{cummins2018compiler} 提出的是一种快速、有效且省力的方法来生成用于编译器模糊测试的随机程序。
\tool{DeepSmith}使用深度学习来自动构建人类如何编写代码的概率模型,它能够推断出编程语言的语法和语义以及常见的结构和模式。
其方法的本质是将随机程序的生成框定为语言建模问题,这极大的简化了并加速了生成过程。
如此以来,生成程序的表现力仅受语料库中包含的内容的限制,不需要开发人员的专业知识或时间消耗。

上述研究表明,将模糊测试应用于无人机当是发现系统设计问题的切实有效的方法。
且与预测模型的结合行之有效。




\subsection{无人机系统分析与修复}
近年来的研究,通过分析无人机系统中的逻辑问题发现了许多漏洞。
从软件设计的角度来看,无人机程序存在一些系统缺陷,可能导致不可预测的结果或在攻击无人机时被利用。
~\citet{wang2021exploratory} 通过调查无人机错误报告、源代码、补丁和历史数据,并对两个主要的无人机开源自动驾驶软件平台,分析了无人机特定错误和相关根本原因,总结了常见的错误模式和修复策略。
\citet{liang2021understanding}进行了一项实证研究,以从自下而上的现场开发人员的角度理解软件的安全关键问题。
他们专注于控制软件中边界函数(Boundary Function)的运行时检查,使用静态方法对边界函数实例进行分类,并通过差异方法动态评估边界函数使用的影响。
\tool{RVFuzzer}~\cite{rvfuzzer}探讨了无人机控制程序中的一种新型漏洞,它涉及对控制参数输入的缺失或不正确的验证检查。
它通过控制引导的输入突变查找无人机控制程序中的输入验证错误。
\tool{RVFuzzer}涉及一个控制不稳定性检测器,该检测器通过基于控制模型观察(模拟)无人机的物理操作来检测控制程序的不当行为。 
此外,它还通过利用控制不稳定性检测器的结果作为反馈,引导输入生成以更有效地查找输入验证错误。
\citet{choi2020cyber} 提出了一种新的基于测试的网络物理不一致(Cyber-Physical Inconsistencies)性检测技术。
该程序分析并提取无人机关于状态或传感器的安全检测模块的谓词条件。
它对该谓词条件进行测试,并通过虚拟车辆和物理测试车辆的差异化状态来测试出无人机运行时物理状态和网络空间不一致的漏洞。
这种的漏洞可能导致安全检查无法报告真实的物理事故或误报。

此外,针对无人机出现安全问题或者系统问题后的修复方式,以下方案也进行了探讨。
随着无人机的广泛采用,它们的事故越来越多,需要对此类事故进行深入调查。
\tool{MAYDAY}~\cite{mayday}认为当前无人机的飞行日志仅记录高级车辆控制状态和事件,不记录控制程序执行。
这使得控制器异常到程序错误定位之间缺乏有效的追踪。
而\tool{MAYDAY},旨在通过将控制模型映射到控制程序来实现无人机事故的跨域调查。
它分析控制和程序级日志从而追踪导致事故的控制语义错误,给出无人机事故的错误原因。
\citet{choi2020software}为多传感器载具提出了一种新的软件传感器恢复技术以抵御物理传感器的攻击。
该方法采用所谓的软件传感器作为相应物理传感器的备份,它可以在同类或不同类的多个传感器受到攻击时恢复。
此方法是纯粹基于软件的,不需要任何额外的硬件。
在运行过程中,软件传感器不断计算和预测相应物理传感器的读数。
当检测到某些物理传感器受到攻击时,相应的软件传感器可以隔离和替换被攻击的传感器,并从损坏的内部状态中恢复系统,以防止严重的攻击后果(例如,崩溃和物理损坏)。
与上述方法类似,\tool{PID-Piper}~\cite{dash2021pid}是一个通过使用辅助控制器与无人机的主控制器串联来实现运行时恢复的框架。
不同于\tool{MAYDAY},\tool{PID-Piper}用前馈控制器,而不是后馈控制器。
使用前馈控制器可以有效避免避免控制器的过度补偿问题。
其次,\tool{PID-Piper}使用机器学习来学习预测控制器输出的模型,并使用特征工程来避免模型的高共线性问题。
\tool{PGPATCH}~\cite{kim2022pgpatch} 为无人机车辆控制程序提出了一个策略引导的程序修复框架,它可以为给定的逻辑错误生成补丁并在没有人为干预的情况下应用它们。
\tool{PGPATCH}由四个相互关联的组件组成:(1) 预处理器,(2) 补丁类型分析器,(3) 补丁生成器,以及 (4) 补丁验证器。
\tool{PGPATCH}首先使用预处理器接受一个触发错误的输入,然后通过补丁类型分析器找到最合适的补丁类型来修复特定的逻辑错误,进而通过补丁生成器组件为目标错误找到合适的补丁位置,并创建一个补丁,最终通过补丁验证器确认该补丁没有破坏无人机的功能或降低其性能。




\section{研究内容}
为了实现无人机系统可靠性,本文从无人机系统本身出发,针对上述无人机系统中仍存在的系统问题,研究更加可靠的安全检查、参数配置和在线修复方案。
具体研究内容如下:
\begin{itemize}
    \item \textbf{不变量验证的可靠安全检查方案:} 
    由于传统的安全检查模块在处理复杂不安全事件上能力有限,所以本文提出了一种新的可靠安全检查方案。
    通过深度学习模型引入不变量异常检测系统,其核心是生成了一种状态不变量,这是一种特殊的预测器,能够描述未来的飞行状态变化。
    系统结合预测偏离检测的方式生成对当前飞行的状态描述。
    然后,该模块使用基于机器学习的检测器来识别是否发生了复杂事件,并制定出相应的应对策略。
    这种新的安全检查模块能够识别出无人机自身系统检查未能检测到的遗漏事件,并减少由于原先检测系统过于敏感而产生的误报。

    
    \item \textbf{模型预测的可靠参数配置方案:} 
    由于无人机系统面临范围规范错误问题, 本文提出一种模型预测的可靠参数配置方案。
    该方案针对无人机系统参数配置机制中的范围规范错误问题,使用预测引导模糊测试的思路,识别出可能引发错误的配置或不合适的参数组合,并对这些参数提出更合理的范围。
    为了更精确地找到可能导致无人机出现范围规范错误的配置,该方案采用了元启发式搜索算法。
    而且,为了提高配置评估的速度,该方案还引入了基于机器学习的预测器。
    这种方法能有效降低模糊测试中的执行验证次数,从而大大提升了搜索的效率。
    其能有效减少参数配置对于无人机系统可靠性的影响。
    
    
    \item \textbf{强化学习的可靠在线修复方案:} 
    针对无人机姿态稳定算法无法应对不正确配置的影响(攻击或自身设置错误),本文提出一种强化学习的可靠在线修复方案。
    该方案的核心是通过观察偏差数据,即实际飞行状态与预期飞行状态之间的差距,来预测潜在的不稳定状态。一旦预测到不稳定状态,该方案会在无人机失控前对其进行纠正。
    在纠正不稳定状态时,该方案会调用预先通过强化学习训练的智能代理,自动生成适当的修正,然后重新配置无人机以避免进入失控状态。
    这种修正过程是迭代进行的,直到不稳定性被消除。
    其能有效减少在线飞行时受到攻击而对系统可靠性产生影响。
    
\end{itemize}

% 本文的主要贡献如下所示:
% \begin{enumerate}
%     \item 本文提出一个用于安全检查模块的辅助模块,\deccheck。 
%     其为不安全事件创建特征描述,并利用检测器识别这些事件,同时提供适当的建议。
%     具体来说,受不变性属性的启发,本文设计一种为无人机提取状态不变的新方法。
%     不变量可以估计未来的飞行状态,并为复杂的不安全事件创建更精确的特征描述。
%     \deccheck 通过结合这些特征与一个神经网络分类器,识别飞行中出现检测模块的错误汇报。
%     提高了无人机自身检查模块的可靠性,降低潜在不安全事件对于无人机的影响。
    
%     \item 本文提出一种模糊测试模块结合预测器来寻找不正确的配置并提供指导,\icsearcher。 
%     \icsearcher 依赖于遗传算法(GA)和飞行状态预测器来检测潜在的不正确配置。
%     它利用过往的飞行经验来生成一个预测器,用于在模糊测试过程中评估潜在不正确配置。
%      同时,\icsearcher 使用段偏差数据来量化配置的效果,以减轻瞬态偏差或噪声的影响。
%      为了平衡开发人员要求的可靠性和适应性,\icsearcher 系统利用优化方法来提供多种灵活的范围指南,以最大限度地减少包括错误配置的可能性,同时保持广泛的范围空间。
%      提高了无人机在使用过程中的可靠性,保障用户在使用过程中的安全性。

%     \item 本文提出一种飞行状态矫正模块,\nyctea。
%     \nyctea 结合配置设置和强化学习的特点来构造一个自适应的智能代理。
%     通过综合探索各个飞控程序中配置与对应飞行状态之间的相关性,代理可以根据物理飞行状态和当前操作要求生成适当的配置。
%     \nyctea 建立在大多数飞行控制程序支持的配置设置方案之上,它可以很容易地与现有的飞行控制程序集成。
%     \nyctea 搞笑准确的解决了无人机飞行过程中的动态校准问题,在发现异常的同时,能够在不中断任务的情况下有效的提高无人机飞行的可靠性。
    
% \end{enumerate}
  
\section{本文组织结构}
本文主要研究面向无人机的安全性和可靠性强化方法。
全文分为六个章节,各章节内容如下:

\textbf{第一章}为绪论,
\textbf{第二章}为预备知识,对本文后续用到的关键技术进行介绍。
\textbf{第三章}对于无人机安全检查模块中的存在的检查不充分的情况进行了分析。有针对性的设立了辅助检测工具用于发现这类检查不充分的情况和场景,并提供更更加合理的应对措施。
\textbf{第四章}针对无人机系统自身配置设置存在的功能缺陷问题,设计了基于学习引导的模糊测试方案。用于发现无人机参数配置中的潜在不正确配置,这些配置可能会导致飞行安全问题。最后通过多目标优化手段,初步提供更加安全的范围指导。
\textbf{第五章}针对无人机飞行在线场景受到内部或外部干扰的情况下,使用在线检测和实时修复的方式避免了无人机不稳定的飞行。
\textbf{第六章}对本文的研究进行了总结和展望。