\section{本文工作总结}
本工作从三个角度探讨了无人机安全性和可靠性的增强方法进行了讨论。

首先,文章从无人机系统的安全检查机制入手,用来解决飞行过程中由\emph{不足事件}和\emph{过度事件}引起的模块检查错误。
\deccheck 模块使用状态不变量和机器学习分类方法来解决上述事件。
\deccheck 建立了收集各种事件数据的环境,利用神经网络来提取飞行状态的不变量。
\deccheck 应用该不变因素为每个事件创建相应的不变差异特征,并训练一个\tool{CNN}模型来检测它们。
本文在模拟和真实的无人机上评估了该系统。
不变量提供了超过98\%的预测精度,每个类别的分类器的F1分数都超过90\%。
与原来的安全检查相比,\deccheck 增强了对特殊物理事件的检测能力。

针对无人机系统控制本身,本文通过\icsearcher 发现了\emph{范围规范错误},该错误是由合法用户或者攻击者触发的一种系统逻辑问题。
最严重的情况会导致飞行不稳定,中断无人机的正常飞行任务。
同时,为了应对这种问题,文章通过设计\icsearcher 模块这样一个基于模糊测试的系统来效地搜索潜在的不正确参数配置。
\icsearcher 使用启发式搜索的手段进行配置搜素,同时通过使用机器学习的预测模型来辅助模糊搜索过程。
最终,通过多目标优化,\icsearcher 实现了一种简单的建议范围的生成方法,初步得解决了这种配置漏洞带来的安全问题,降低了无人机飞行过程中出现不稳定状态的概率。
本章将该工具与其他先进研究进行比较,表明了\icsearcher 自身系统的优越性。

在初步讨论了\emph{范围规范错误}所带来的问题后,文章针对范围限制的这种补救形式的不足,进一步的提出了在线修复的思路。
\nyctea 可以在无人机飞行过程中高效地检测出由\emph{范围规范错误}引起的不稳定飞行状态,同时提供更加合适的整改方案。
再确定了无人机正处于危机时刻的时候,\nyctea 使用预先训练的强化学习智能体来进行迭代矫正,指导无人机稳定飞行。
本文通过实验测试进一步论证了\nyctea 模块的修复效率和成功率。


\section{研究展望}
本文探讨了三种无人机系统安全与可靠性增强机制,后续计划研究如下:
\begin{enumerate}
    \item 进行适配迁移,将\deccheck 应用于更多无人机操作系统当中。同时引入自动化的方法,更加精准的采集无任务发生安全事件后的数据变化模式,提高系统在对事件进行识别时的准确性,降低不同类别之间数据分类的干扰,降低误报率。

    \item 由于Range Specification Bugs是通过Fuzzing确定的,目前尚不清楚其根本性原理。
    计划采用动态静态相结合的方法,精准定位设计的缺陷之处,在无人机汇报安全问题的同时,精准快速定位参数问题的根本所在,以最小的修改代价稳定无人机的飞行姿态,确保无人机的飞行任务顺利完成。
    
\end{enumerate}